\chapter{Introduction}
\label{ch:intro}
\thispagestyle{empty}
\pagenumbering{arabic}
\setcounter{page}{1}


The idea that we learn by interacting with our environment is probably the first to occur to us when we think about the nature of learning. When an infant plays, waves its arms, or looks about, it has no explicit teacher, but it does have a direct connection to its environment. Exercising this connection produces a wealth of information about cause and effect, about the consequences of actions, and about what to do in order to achieve goals. Throughout our lives, such interactions are undoubtedly a major source
of knowledge about our environment and ourselves. Whether we are learning to drive a car or to hold a conversation, we are acutely aware of how our environment responds to what we do, and we seek to influence what happens through our behavior. Learning from interaction is a foundational idea underlying nearly all theories of learning and intelligence.

\section{Inquadramento generale}
La prima parte contiene una frase che spiega l'area generale dove si svolge il lavoro; una che spiega la sottoarea pi\`u specifica dove si svolge il lavoro e la terza, che dovrebbe cominciare con le seguenti parole ``lo scopo della tesi \`e \dots'', illustra l'obbiettivo del lavoro. Poi vi devono essere una o due frasi che contengano una breve spiegazione di cosa e come \`e stato fatto, delle attivit\`a sperimentali, dei risultati ottenuti con una valutazione e degli sviluppi futuri. La prima parte deve essere circa una facciata e mezza o due

\section{Breve descrizione del lavoro}
La seconda parte deve essere una esplosione della prima e deve quindi mostrare in maniera pi\`u esplicita l'area dove si svolge il lavoro, le fonti bibliografiche pi\`u importanti su cui si fonda il lavoro in maniera sintetica (una pagina) evidenziando i lavori in letteratura che presentano attinenza con il lavoro affrontato in modo da mostrare da dove e perch\'e \`e sorta la tematica di studio. Poi si mostrano esplicitamente le realizzazioni, le direttive future di ricerca, quali sono i problemi aperti e quali quelli affrontati e si ripete lo scopo della tesi. Questa parte deve essere piena (ma non grondante come la sezione due) di citazioni bibliografiche e deve essere lunga circa 4 facciate.

\section{Struttura della tesi}
La terza parte contiene la descrizione della struttura della tesi ed \`e organizzata nel modo seguente.
``La tesi \`e strutturata nel modo seguente.

Nella sezione due si mostra \dots

Nella sez. tre si illustra \dots

Nella sez. quattro si descrive \dots

Nelle conclusioni si riassumono gli scopi, le valutazioni di questi e le prospettive future \dots

Nell'appendice A si riporta \dots (Dopo ogni sezione o appendice ci vuole un punto).''

I titoli delle sezioni da 2 a M-1 sono indicativi, ma bisogna cercare di mantenere un significato equipollente nel caso si vogliano cambiare. Queste sezioni possono contenere eventuali sottosezioni.
